\documentclass[12pt]{article}
\usepackage{geometry}
\geometry{a4paper,left=2cm,right=2cm,top=1cm,bottom=2cm}
\usepackage[utf8]{inputenc}
\usepackage{natbib}
\usepackage{bm}
\usepackage[colorlinks]{hyperref}
\usepackage{color}
\usepackage{graphicx}
\usepackage{amsmath}
\usepackage{amssymb}
\newcommand{\dd}{{\rm d}}

\title{Homework 1}
\author{Yong Gao \\ gaoyong.physics@pku.edu.cn}
\date{ }

\begin{document}
\maketitle

\section{Problem 2}
Read or re-read four papers, preferably with at least one that you remember to
be well written and one that you had trouble understanding.
\subsection{Four papers}
{\color{blue}{[1] Tanja Hinderer. Tidal love numbers of neutron stars. ApJ, 677(2):1216–1220, 2008.}}
\newline
{\color{blue}{[2] Theocharis A. Apostolatos, Curt Cutler, Gerald J. Sussman, and Kip S. Thorne.  Spininduced orbital precession and its modulation of the gravitational waveforms from merging binaries.Phys. Rev. D, 49:6274–6297, 1994. }}
\newline
{\color{blue}{[3] Ken Nordtvedt.  Probing gravity to the second post-Newtonian order and to one part in $10^{7}$ using the spin axis of the Sun.ApJ, 320:871, 1987.}}
\newline
{\color{red}{[4] Bo Wang and Dongdong Liu.  The formation of neutron star systems through accretion-induced collapse in white-dwarf binaries.RAA, 20(9):135, 2020.}}
\newline
\newline
I think the first three in blue are good papers, while the last one in red is a bad one.
\subsection{Good papers}
\textbf{For the first paper}, I like the style that the author organized the equations, figures and tables to explain the physics of tidally-deformed neutron stars. Here are reasons why I think it is well written,
\begin{itemize}
    \item The equations are clear and concise. The author organize lots of equations with powerful logic and ignore detail procedures.
    \item Explain the physics under these equations with the help of figures and tables. 
    \item The author give the most important results briefly in the abstract.
\end{itemize}
\textbf{For the second paper}, I like the organization of the whole paper and the beautiful figures. Here are reasons why I think it is well written,
\begin{itemize}
    \item The organization of this paper is friendly to the readers. The authors explained the precession of binary system step by step, from non-precessing binaries to precessing ones and some special cases to give intuitions. Finally it goes to the general cases. It is a pleasure to read this paper. I can understand step by step. It's like the authors teach me how to understand something.
    \item The figures are beautiful and useful. The main goal of this paper is to develop intuition into what the precessing waveforms “look like” with reasonable assumptions and simplifications. These figures helped a lot.
    \item First they gave order of magnitude estimations. Then it come to the detailed calculations.
\end{itemize}
\textbf{For the third paper}, I like the title. Here are reasons why I think it is well written,
\begin{itemize}
    \item The title is "Probing gravity to the second post-Newtonian order and to one part in $10^7$ using the spin axis of the Sun". The author gave the main results and the method directly. I am attracted by this title when I first saw it.
    \item The paper is concise and clear. 
    \item Every assumption and approximation were given.
\end{itemize}

\subsection{Bad paper}
The authors wanted to convey lots of information to the readers, but without a clear writing and good logic. Besides, as a review paper, the authors tried to include many things but without main points. Here are some examples,
\begin{itemize}
    \item The abstract is too long and it nearly contains every results. I think it is unnecessary. 
    \item Clutter in the abstract. 
        \begin{enumerate}
            \item {\color{red}{\emph{"However, there has been no direct detection of any AIC event so far, even though there exists a lot of indirect
            observational evidence}".}} The latter part can be deleted.
            \item {\color{red}{\emph{"In this article, we review recent studies on the two classic progenitor models of AIC events, i.e., the single-degenerate model (including the ONe WD+MS/RG/He star channels
            and the COWD+He star channel) and the double-degenerate model (including the double COWD channel,
            the double ONe WD channel and the ONe WD+CO WD channel)}".}} I think the details of the two models can be deleted.
        \end{enumerate}
    \item Clutter in the introduction. 
        \begin{enumerate}
            \item In the fourth paragraph of the introduction, the authors listed the troublesome of AIC process. As they have talked that the list is about AIC, it is unnecessary to type AIC every time. Same problems also appear in many other places in the paper.
        \end{enumerate}
    \item Just outline different channels without much comparison. The figures for every channel are similar and I think they could organize better and make some merges.
\end{itemize}
\section{Problem 3; Reverse outlining the introduction}
I choose the first paper to do reverse outlining. Here are brief ideas for every paragraph,
\begin{enumerate}
    \item The interior of neutron stars are unclear
    \item Gravitational waves can probe 
    \item Gravitational wave signal includes inspiral and merger
    \item Inspiral give tidal corrections, a good and clean probe
    \item Neutron stars are complex compared to Newtonian stars
    \item Analytical methods are useful for comparing different numerical solutions
    \item Use polytropic model to calculation tidal effects and the structures of the paper
\end{enumerate}

\quad This logic is very good. First the author gave the main problem: the equation of state of neutron stars are unclear and it is a fundamental physics problem. Second, the author told us that gravitational waves from binary neutron star mergers can constrain the equation of state. Third, the author discussed the different stages of gravitational wave radiation from binary neutron stars and told us that the merger phase is too complex. Fourth, the author said that the tidal corrections in the inspiral phase are much more clean and can be possibly detected by future LIGO detectors. Next, the author stressed that the main challenge of the calculations of the tidal corrections and told us the main methods and results. Finally, the author outlined the structure of this paper.
\newline

\quad This logic gives the main problem and tells us how to crack it.
\end{document}
